% --------------------------------------------------------------
% This is all preamble stuff that you don't have to worry about.
% Head down to where it says "Start here"
% --------------------------------------------------------------
 
\documentclass[12pt]{article}
 
\usepackage[margin=1in]{geometry} 
\usepackage{amsmath,amsthm,amssymb}
\usepackage{graphicx}
\usepackage{enumitem}
\usepackage{pifont}
\usepackage{enumerate}
\usepackage{xcolor}
\definecolor{smithblue}{HTML}{002855}
\definecolor{smithyellow}{HTML}{F2A900}

\usepackage[parfill]{parskip}
\parskip=\baselineskip
 
\newcommand{\N}{\mathbb{N}}
\newcommand{\Z}{\mathbb{Z}}
 

\newenvironment{exercise}[2][Exercise]{\begin{trivlist}
\item[\hskip \labelsep {\bfseries #1}\hskip \labelsep {\bfseries #2.}]}{\end{trivlist}}

\newenvironment{solution}[1][{\color{red} Solution:}]{\begin{trivlist}
\item[\hskip \labelsep {\bfseries #1}\hskip \labelsep {\bfseries}]}{\end{trivlist}}


\usepackage{fancyhdr}
\pagestyle{fancy}
\lhead{Submitted by: \studentName\\
\collaborators}
\rhead{CSC250 Spring 2023 - Homework 02\\
\today{}}
\cfoot{p. \thepage}
\renewcommand{\headrulewidth}{0.4pt}
\renewcommand{\footrulewidth}{0.4pt}

\begin{document}
 
% --------------------------------------------------------------
%                         Start here
% --------------------------------------------------------------


\newcommand{\studentName}{Sabrina Hatch} %replace with your name

\newcommand{\collaborators}{ 
	% Comment out the line below if you worked alone
	with \textit{Sabrina Hatch, Ramsha Rauf, Shalom Mhanda}
	% Uncomment the line below if you worked alone
	% \textit{I did not collaborate with anyone on this assignment.}
}

% --------------
% Exercise 1
% --------------
\begin{exercise}{1}
For each of the following REs on the alphabet $\Sigma = [a,b,c]$, identify one word that is \textbf{in the language} recognized by the RE, and one word that is \textbf{not in the language}.
\begin{enumerate}[(a)]
\item $(ab)^*+bc^*$
% -------------------------------------------
%  Write your answer to Q1a below
% -------------------------------------------
\begin{solution} 
\quad\\
IN: abab\\
OUT: bababa \\ 
\end{solution}

\item $a^*bc^*$
% -------------------------------------------
%  Write your answer to Q1b below
% -------------------------------------------
\begin{solution}
\quad\\
IN:  b \\
OUT:  abbbc
\end{solution}

\item $(a+b)^*(b+c)^*$
% -------------------------------------------
%  Write your answer to Q1c below
% -------------------------------------------
\begin{solution}
\quad\\
IN:  ab \\
OUT:  cabb
\end{solution}

\end{enumerate}
\end{exercise}

\clearpage

% --------------
% Exercise 2
% --------------
\begin{exercise}{2}

Give a regular expression for each of the following languages. In all cases, the alphabet is $\Sigma = \{0,1\}$.
\begin{enumerate}[(a)]
	\item $\{ w \in \Sigma^* \ | \ w \texttt{ has an odd number of 0s followed by a single 1}\}$
	% -------------------------------------------
	%  Write your answer to Q2a below
	% -------------------------------------------
	\begin{solution}
 $0(00)^*1$
	\end{solution}
	
	\item $\{w \in \Sigma^* \ | \ w \texttt{ starts in a triple letter (000 or 111)}\}$
	% -------------------------------------------
	%  Write your answer to Q2b below
	% -------------------------------------------
	\begin{solution}
 $(000 + 111) \Sigma^*$
	\end{solution}
	
	\item $\{w \in \Sigma^* \ | \ w \texttt{ contains exactly three 1s or at least two 0s}\}$
	% -------------------------------------------
	%  Write your answer to Q2c below
	% -------------------------------------------
	\begin{solution}
 $(0^* \ 1 \ 0^* \ 1 \ 0^* \ 1 \  0^*)  + (\Sigma^* \ 0  \ \Sigma^* \ 0 \ \Sigma^* $)
	\end{solution}
	
\end{enumerate}

\end{exercise}

\clearpage
% --------------
% Exercise 3
% --------------
\begin{exercise}{3}
Show that any language $L_F$ containing only finitely many strings is regular.
\end{exercise}

% -------------------------------------------
%  Write your answer to Q3 below
% -------------------------------------------
\begin{solution}
 Any language containing finitely many strings is regular because we can generate the language using a regular expression with a finite amount of alternation. For example, if we have $L_F$ with k number of strings, then we can generate the language using a regular expression with k-1 operations of an alternation.



\begin{proof}[\unskip\nopunct]
   \qedhere
\end{proof}

\end{solution}


\clearpage

% --------------
% Exercise 5\4
% --------------
\begin{exercise}{4}

Show that if $L$ is a regular language, then the reverse language $L^R$:

$$L^R = \{w^R \ | \ w\in L \texttt{ and } w^R \texttt{ is the word } w \texttt{ written in reverse}\}$$

is also a regular language (i.e. regular languages are closed under \textbf{reversal}).

Remember, proving a language is regular is all about proving it has a regular expression that can generate it!

{\Large \textit{Hint: }}
 try induction! by:
\begin{enumerate}
    \item \textbf{Base cases}: prove this works for the 3 base cases:
    \begin{itemize}
        \item $L =\emptyset$ (show the reverse of an empty language is regular)
        \item $w=\epsilon$ (show you can get the regular expression for the reverse of an empty word)
        \item $w=a$ \text{ where} $a$ \text{ represents any single valid symbol}(show you can get the regular expression for the reverse of a word made of a single symbol)
    \end{itemize}
    \item\textbf{ Induction Hypothesis}: assume that you have 
    \begin{itemize}
        \item two regular expressions: $E$ and $F$ that generate $L_F = L(F)$ and $L_G = L(G)$ respectively, AND for which
        \item we are able to find a regular expressions $F^R$ and $G^R$ such that  $L^R_F = L(F^R)$ and $L^R_G = L(G^R)$, and where $L^R_F = \text{ Reverse of } L_F$ and $L^R_G = \text{ Reverse of } L_G$
    \end{itemize}

    \item \textbf{Induction Step}: Using the hypothesis, show how we can reverse all of the operations for extending regular expressions (alternation, concatenation, and Kleene star); 
\end{enumerate}

\end{exercise}

% -------------------------------------------
%  Write your answer to Q4 below
% -------------------------------------------
\begin{solution}
 \item To start, we need to prove the base cases. 
     \begin{itemize}[label=\ding{212}]
     \item First, we will prove this for the reverse of an empty language $L = \emptyset$. 
        \begin{itemize}[label=\ding{75}]
        \item The regular expression for this empty language is $RE = \emptyset$. The reverse of  $RE$, denoted by $RE^R$, is also equal to $\emptyset$. Therefore, we can represent the language generated by $RE^R$ as $L^R = \emptyset$. And since the empty set can be described by a regular expression, $L^R = \emptyset$ is also regular. 
        \end{itemize}
     \item Next, we will prove this for the reverse of the language containing only the empty word $L=\epsilon$.
        \begin{itemize}[label=\ding{75}]
        \item The regular expression for this language containing only the empty word is $RE = \epsilon$. The reverse of  $RE$, denoted by $RE^R$, is also equal to $\epsilon$. Therefore, we can represent the word generated by $RE^R$ as $L^R = \epsilon$. And since epsilon can be described by a regular expression, $L^R = \epsilon$ is also regular. 
        \end{itemize}
     \item Finally, we will prove this for the reverse of the language containing only one word composed of a single symbol $L=a$.
     \begin{itemize}[label=\ding{75}]
        \item The regular expression for this language containing only a single symbol is $RE = a$. The reverse of  $RE$, denoted by $RE^R$, is also equal to $a$. Therefore, we can represent the word generated by $RE^R$ as $L^R = a$. And since a symbol can be described by a regular expression, $L^R = a$ is also regular. 
     \end{itemize}
  \item Now, we craft our induction hypothesis. 
  \begin{itemize}[label=\ding{75}]
        \item Assuming that you have a regular language that is represented by $L(F)$, where F is the regular expression generating the language. If you reverse $L(F)$, the result is another regular language denoted by $L(F^R)$, where $F^R$ is a regular expression generating the reversed language $L(F^R)$.
     \end{itemize}

\item Now for the induction step, we will show how we can reverse  all of the operations for extending regular expressions (alternation, concatenation, and Kleene star). 
     \begin{itemize}[label=\ding{75}]
        \item If you are given regular expressions $F$ and  $G$, you can construct a regular expression $H$ as an alternation of $F$ and $G$: $H = G + F$. This is equal to $L(H) = L(F) \cup L(G)$. To find the reverse of $L(H)$, represented by $L(H^R)$, we can use our induction hypothesis. Using our induction hypothesis we find $L(H^R) = L(F^R + G^R)$, where $F^R$ and $G^R$ are also regular expressions. Thus the alternation of these two reversed regular expressions will produce a regular reversed language $L(H^R)$. Such that $L(H^R) = L(F^R) \cup L(G^R)$. 
        \end{itemize}

        \begin{itemize}[label=\ding{75}]
        \item If you are given regular expressions $F$ and  $G$, you can construct a regular expression $H$ as a concatenation of $F$ and $G$: $H = FG$. This is equal to $L(H) = L(F) \cap L(G)$. To find the reverse of $L(H)$, represented by $L(H^R)$, we can use our induction hypothesis. Using our induction hypothesis we find $L(H^R) = L(G^RF^R)$, where $F^R$ and $G^R$ are also regular expressions (note that the order of concatenation of $FG$ must also reversed). Thus the concatenation of these two reversed regular expressions will produce a regular reversed language $L(H^R)$. Such that $L(H^R) = L(G^RF^R)$.
        \end{itemize}

        \begin{itemize}[label=\ding{75}]
        \item If you are given regular expression $F$, you can construct a regular expression $H$ as a Kleene star of $F$: $H = F^*$. This is equal to $L(H) = L(F^*)$. To find the reverse of $L(H)$, represented by $L(H^R)$, we can use our induction hypothesis. Using our induction hypothesis we find $L(H^R) = L(F^*)^R = (L(F)^*)^R = L(F^R)^*$, where $F^R$ is also a regular expression. Thus the Kleene star of the reversed regular expression will produce a regular reversed language $L(H^R)$. Such that $L(H^R) = L(F^R)^*$.
        \end{itemize}

\item Therefore the induction hypothesis is true under all 3 operations: alternation, concatenation and Kleene star as the language is closed under all 3 operations.  Thus  the reversal of a regular language $L$ to $L^R$ is also a regular language because regular languages are also closed under reversal. 

     \end{itemize}
\begin{proof}[\unskip\nopunct]
   \qedhere
\end{proof}   
\end{solution}


% -----------------
% References
% -----------------
\vfill
\begin{thebibliography}{9}
\bibitem{sipser} 
Sipser, Michael. 
\textit{Introduction to the Theory of Computation.}
Course Technology, 2005. ISBN: 9780534950972

\bibitem{critchlow2011foundation}
Critchlow, Carol and Eck, David
\textit{Foundation of computation.},
Critchlow Carol, 2011

\end{thebibliography}

% --------------------------------------------------------------
%     You don't have to mess with anything below this line.
% --------------------------------------------------------------
 
\end{document}
